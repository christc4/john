\renewcommand{\footnotesize}{\fontsize{6pt}{7pt}\selectfont}
\renewcommand{\thefootnote}{\roman{footnote}}
\pagestyle{headings}
\markright{The Gospel According to St. John - Patristic Commentary}
\newcommand{\vn}[1]{\textsuperscript{\textcolor{red}{\textbf{#1}}}\hspace{0.15em}\ignorespaces}
%\newcommand{\sn}[1]{\marginpar{\raggedright\footnotesize #1}}

\newcommand{\sn}[1]{\marginpar{\raggedright\footnotesize #1}}
\newcommand{\rn}[1]{\marginpar{\raggedleft\footnotesize #1}}


%\newcommand{\fnm}[1]{\footnotemark{}}
\newcommand{\fnt}[1]{\footnotetext{}}
\newcommand{\fnm}[1]{\footnotemark[#1]}
\setlength{\columnsep}{1cm}

\begin{paracol}{2}
\section*{JOHN}
    \lettrine{I}{n} the beginning\fnm{1} was\sn{Recalls Genesis... (Greek Septuagint) ``\textit{In the beginning God created...}
Genesis, speaks of first creation in Adam, who brought sin and death into the world, Christ is the second Adam, new creation in him, who brings redemption and new life.

To see how ``Beginning'' can refer to see Proverbs
    } the Word\fnm{2}, and the Word was with\fnm{3} God, and the Word was\fnm{4} God.\vn{2}He was in the beginning with God.\vn{3}All things\footnotemark[5] were made through Him, and without Him nothing was made that was made.
\switchcolumn
%Find the dabar, hebrew passage

\section*{ΙΩΑΝΝΗΝ}
    \lettrine[ante=']{E}{ν ἀρχῇ} ἦν ὁ λόγος\footnotemark[2], καὶ ὁ λόγος\rn{ \cjRL{dabar}
        wide semantic domain, rational principle (Philo) logos, (Logos is analogous to the Wyrd, Way, Tao, Law etc.
    Greek: “word,” “reason,” or “plan”) plural logoi, in ancient Greek philosophy and early Christian theology, the divine reason implicit in the cosmos, ordering it and giving it form and meaning.} ἦν πρὸς\footnotemark[3] τὸν θεόν, καὶ θεὸς ἦν\footnotemark[4] ὁ λόγος.\vn{2}οὗτος ἦν ἐν ἀρχῇ πρὸς τὸν θεόν.\vn{3}πάντα\footnotemark[5] δι’ αὐτοῦ ἐγένετο, καὶ χωρὶς αὐτοῦ ἐγένετο οὐδὲ ἕν. ὃ γέγονεν.
\end{paracol}


\footnotetext[1]{beginning - (gr. \textit{ἀρχῇ})\\
\textit{\textbf{St. Basil the Great, Hexameron}} - ``The beginning, in effect, is indivisible and instantaneous. The beginning of the road is not yet the road, and that of the house is not yet the house; so the beginning of time is not yet time and not even the least particle of it. If some objector tell us that the beginning is a time, he ought then, as he knows well, to submit it to the division of time — a beginning, a middle and an end. Now it is ridiculous to imagine a beginning of a beginning. Further, if we divide the beginning into two, we make two instead of one, or rather make several, we really make an infinity, for all that which is divided is divisible to the infinite...''\\
\textit{\textbf{St. John Chrysostom, Homily 2}} - ``For the intellect, having ascended to \textit{``the beginning''}, enquires what \textit{``beginning''}; and then finding the \textit{``was''} always outstripping its imagination, has no point at which to stay its thought; but looking intently onwards, and being unable to cease at any point, it becomes wearied out, and turns back to things below. For this \textit{``was in the beginning,''} is nothing else than expressive of ever being and being infinitely. As Paul also declared, when he said, Having neither beginning of days, nor end of life Hebrews 7:3; by this expression showing that He is both without beginning and without end. For as the one has no limit, so neither has the other. In one direction there is no end, in the other no beginning.''\\
\textit{\textbf{St. Cyril of Alexandria, Homily 2}} - ``
There will then be no beginning of beginning, according to exact and true reasoning, but the account of it will recede unto the long-extended and incomprehensive. And 12 since its ever-backward flight has no terminus, and reaches up to the limit of the ages, the Son will be found to have been not made in time, but rather invisibly existing with the Father: for in the beginning was He. But if He was in the beginning, what mind, tell me, can over-leap the force of the was? When will the was stay as at its terminus, seeing that it ever runs before the pursuing reasoning, and springs forward before the conception that follows it?}

\footnotetext[2]{Word - (gr. \textit{Λόγος})\\
\textit{\textbf{St. Justin Martyr, The First Apology}} -
``Himself, who took shape, and became man, and was called Jesus Christ''}
\footnotetext[3]{was - (gr. \textit{ἦν})}
\footnotetext[3]{with - (gr. \textit{πρὸς})}
\footnotetext[4]{all things - (gr. \textit{πάντα})\\
\textit{\textbf{St. Irenaeus, Against Heresies (Book II, Chapter 2)}} - ``Now, among the `\textit{all things}' our world must be embraced. It too, therefore, was made by His Word, as Scripture tells us in the book of Genesis that He made all things connected with our world by His Word. David also expresses the same truth [when he says] `For He spoke, and they were made; He commanded, and they were created'.''}

\newpage

\begin{paracol}{2}
    \vn{4}In Him was life\fnm{1}, and the life\sn{God has life in himself, source of life.} was the light\fnm{2} of men.\vn{5}And the light\sn{testing} shines in the darkness, and the darkness did not comprehend\sn{κατέλαβεν - comprehend can mean ``understand'' and ``overcome'', darkness can neither overpower the light of christ, nor can it understand the way of love.}I it.
\switchcolumn
 \vn{4}ἐν αὐτῷ ζωὴ\fnm{1} ἦν, καὶ ἡ ζωὴ ἦν τὸ φῶς\fnm{2} τῶν ἀνθρώπων·\vn{5}καὶ τὸ φῶς ἐν τῇ σκοτίᾳ φαίνει, καὶ ἡ σκοτία αὐτὸ οὐ κατέλαβεν.
\end{paracol}

\footnotetext[1]{life - (gr. \textit{ζωὴ})
\textit{\textbf{St. John Chrysostom}}
For the word Life here refers not merely to the act of creation, but also to the providence (engaged) about the permanence of the things created; it also lays down beforehand the doctrine of the resurrection, and is the beginning of these marvelous good tidings. Since when life has come to be with us, the power of death is dissolved; and when light has shone upon us, there is no longer darkness, but life ever abides within us, and death cannot overcome it.
}
\footnotetext[2]{light - (gr. \textit{φῶς})
\textit{\textbf{St. Augustine of Hippo}}
For this follows: and the life was the light of men; and from this very life are men illuminated. Cattle are not illuminated, because cattle have not rational minds capable of seeing wisdom. But man was made in the image of God, and has a rational mind, by which he can perceive wisdom. That life, then, by which all things were made, is itself the light; yet not the light of every animal, but of men. Wherefore a little after he says, That was the true light, which lights every man that comes into the world. By that light John the Baptist was illuminated; by the same light also was John the Evangelist himself illuminated. He was filled with that light who said, I am not the Christ; but He comes after me, whose shoe's latchet I am not worthy to unloose. John 1:26-27 By that light he had been illuminated who said, In the beginning was the Word, and the Word was with God, and the Word was God. Therefore that life is the light of men.
}
\newpage

\begin{paracol}{2}
    \vn{6}There was a man sent from god, whose name was John.
    \vn{7}This man came for a witness, to bear witness of the Light, that all through him might believe.
    \vn{8}He was not that Light, but was sent to bear witness of that Light.
    \vn{9}That was the true Light which gives light to every man coming into the world.
\switchcolumn
    \vn{4}ἐν αὐτῷ ζωὴ\fnm{1} ἦν, καὶ ἡ ζωὴ ἦν τὸ φῶς\fnm{2} τῶν ἀνθρώπων·
    \vn{5}καὶ τὸ φῶς ἐν τῇ σκοτίᾳ φαίνει, καὶ ἡ σκοτία αὐτὸ οὐ κατέλαβεν.
\end{paracol}

\newpage

\begin{paracol}{2}
    \vn{10}He was in the world, and the world was made through Him, and the world did not know Him.
    \vn{11}He came to His own, and His own did not receive Him.
\switchcolumn
 \vn{4}ἐν αὐτῷ ζωὴ\fnm{1} ἦν, καὶ ἡ ζωὴ ἦν τὸ φῶς\fnm{2} τῶν ἀνθρώπων·\vn{5}καὶ τὸ φῶς ἐν τῇ σκοτίᾳ φαίνει, καὶ ἡ σκοτία αὐτὸ οὐ κατέλαβεν.
\end{paracol}

\newpage

\begin{paracol}{2}
\vn{12}But as many received Him, to them He gave the right to become children of God, to those who believe in His name:\vn{13}who were born, not of blood, nor of the will of the flesh, nor of the will of man, but of God.
\switchcolumn
 \vn{4}ἐν αὐτῷ ζωὴ\fnm{1} ἦν, καὶ ἡ ζωὴ ἦν τὸ φῶς\fnm{2} τῶν ἀνθρώπων·\vn{5}καὶ τὸ φῶς ἐν τῇ σκοτίᾳ φαίνει, καὶ ἡ σκοτία αὐτὸ οὐ κατέλαβεν.
\end{paracol}

\newpage

\begin{paracol}{2}
\vn{14}And the Word became flesh and dwelt among us, and we beheld His glory, the glory as of the only begotten of the Father, full of grace and truth.\vn{15}John bore witness of Him and cried out, saying, ``This was He of whom I said, `He who comes after me is preferred before me, for He was before me.''
\switchcolumn
 \vn{4}ἐν αὐτῷ ζωὴ\fnm{1} ἦν, καὶ ἡ ζωὴ ἦν τὸ φῶς\fnm{2} τῶν ἀνθρώπων·\vn{5}καὶ τὸ φῶς ἐν τῇ σκοτίᾳ φαίνει, καὶ ἡ σκοτία αὐτὸ οὐ κατέλαβεν.
\end{paracol}

\newpage

\begin{paracol}{2}
\vn{16}And of His fullness we hae all received, and grace for grace. For the law was given through Moses, but grace and truth came through Jesus Christ.
\switchcolumn
 \vn{4}ἐν αὐτῷ ζωὴ\fnm{1} ἦν, καὶ ἡ ζωὴ ἦν τὸ φῶς\fnm{2} τῶν ἀνθρώπων·\vn{5}καὶ τὸ φῶς ἐν τῇ σκοτίᾳ φαίνει, καὶ ἡ σκοτία αὐτὸ οὐ κατέλαβεν.
\end{paracol}

\newpage

\begin{paracol}{2}
No one has seen God at any time. The only begotten Son, who is in the bosom of the Father, He has declared Him.
\switchcolumn
 \vn{4}ἐν αὐτῷ ζωὴ\fnm{1} ἦν, καὶ ἡ ζωὴ ἦν τὸ φῶς\fnm{2} τῶν ἀνθρώπων·\vn{5}καὶ τὸ φῶς ἐν τῇ σκοτίᾳ φαίνει, καὶ ἡ σκοτία αὐτὸ οὐ κατέλαβεν.
\end{paracol}


%Septuagint logos, god's utterances, gen1:3, 6:9, 3:9,11 Ps. 32:9,
%god's action zach 5:1-4 ps. 106:20 ps 147:15
%messages of prohets by means of which god communicatred his will to people
% jer 1:4-19, 2:1-7 ezekieal 1:3 amos 3:1, logos is used here as figure of speech designating gods activity or action,

% the greek metaphysical concept of logos in sharp contrast to concept of personal god described philo of alexandria, hellenized Jew, produced synthesis of traditional
