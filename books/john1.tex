\renewcommand{\footnotesize}{\fontsize{6pt}{7pt}\selectfont}
\renewcommand{\thefootnote}{\roman{footnote}}
\pagestyle{headings}
\markright{The Gospel of St. John - Patristic Commentary}
\newcommand{\vn}[1]{\textsuperscript{\textcolor{red}{\textbf{#1}}}\hspace{0.15em}\ignorespaces}
\newcommand{\sn}[1]{\marginpar{\raggedright\footnotesize #1}}
%\newcommand{\fnm}[1]{\footnotemark{}}
\newcommand{\fnt}[1]{\footnotetext{}}
\newcommand{\fnm}[1]{\footnotemark[#1]}
\setlength{\columnsep}{1cm}

\begin{paracol}{2}
\section*{JOHN}
    \lettrine{I}{n} the beginning\fnm{1} was\sn{Beginning, in Greek is used not to signify a fixed starting point, take for instance} the Word\fnm{2}, and the Word\sn{logos, (Greek: “word,” “reason,” or “plan”) plural logoi, in ancient Greek philosophy and early Christian theology, the divine reason implicit in the cosmos, ordering it and giving it form and meaning.
    Greek: “word,” “reason,” or “plan”) plural logoi, in ancient Greek philosophy and early Christian theology, the divine reason implicit in the cosmos, ordering it and giving it form and meaning.}
    was with\fnm{3} God, and the Word was\fnm{4} God.\vn{2}He was in the beginning with God.\vn{3}All things\footnotemark[5] were made through Him, and without Him nothing was made that was made.
\switchcolumn
\section*{ΙΩΑΝΝΗΝ}
\lettrine[ante=']{E}{ν ἀρχῇ} ἦν ὁ λόγος\footnotemark[2], καὶ ὁ λόγος ἦν πρὸς\footnotemark[3] τὸν θεόν, καὶ θεὸς ἦν\footnotemark[4] ὁ λόγος.\vn{2}οὗτος ἦν ἐν ἀρχῇ πρὸς τὸν θεόν.\vn{3}πάντα\footnotemark[5] δι’ αὐτοῦ ἐγένετο, καὶ χωρὶς αὐτοῦ ἐγένετο οὐδὲ ἕν. ὃ γέγονεν.
\end{paracol}

\footnotetext[1]{Beginning - (Gr. \textit{αρχη})\\
\textit{\textbf{St. Basil the Great, Hexameron}} - ``The beginning, in effect, is indivisible and instantaneous. The beginning of the road is not yet the road, and that of the house is not yet the house; so the beginning of time is not yet time and not even the least particle of it. If some objector tell us that the beginning is a time, he ought then, as he knows well, to submit it to the division of time — a beginning, a middle and an end. Now it is ridiculous to imagine a beginning of a beginning. Further, if we divide the beginning into two, we make two instead of one, or rather make several, we really make an infinity, for all that which is divided is divisible to the infinite...''\\
\textit{\textbf{St. John Chrysostom, Homily 2}} - ``For the intellect, having ascended to \textit{``the beginning''}, enquires what \textit{``beginning''}; and then finding the \textit{``was''} always outstripping its imagination, has no point at which to stay its thought; but looking intently onwards, and being unable to cease at any point, it becomes wearied out, and turns back to things below. For this \textit{``was in the beginning,''} is nothing else than expressive of ever being and being infinitely.''\\
\textit{\textbf{St. Cyril of Alexandria, Homily 2}} - ``
There will then be no beginning of beginning, according to exact and true reasoning, but the account of it will recede unto the long-extended and incomprehensive. And 12 since its ever-backward flight has no terminus, and reaches up to the limit of the ages, the Son will be found to have been not made in time, but rather invisibly existing with the Father: for in the beginning was He. But if He was in the beginning, what mind, tell me, can over-leap the force of the was? When will the was stay as at its terminus, seeing that it ever runs before the pursuing reasoning, and springs forward before the conception that follows it?}


\footnotetext[2]{Word - (Gr. \textit{αρχη})\\
\textit{\textbf{St. Justin Martyr, The First Apology}} -
``Himself, who took shape, and became man, and was called Jesus Christ''}

\footnotetext[3]{was - (Gr. \textit{αρχη})}
\footnotetext[3]{with - (Gr. \textit{αρχη})}
\footnotetext[4]{All things - (Gr. \textit{αρχη})\\
\textit{\textbf{St. Irenaeus, Against Heresies (Book II, Chapter 2)}} - ``Now, among the `\textit{all things}' our world must be embraced. It too, therefore, was made by His Word, as Scripture tells us in the book of Genesis that He made all things connected with our world by His Word. David also expresses the same truth [when he says] `For He spoke, and they were made; He commanded, and they were created'.''}
